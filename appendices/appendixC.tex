\chapter{Modelo epiABM}
\section{Parametrización por defecto}\label{appendix:epi_abm_default_params}

Aquí especificamos la parametrización por defecto que utilizamos para el modelo epiABM. El Cuadro \ref{table:default_params} sintetiza estos valores que fueron elegidos para describir la etapa inicial de la pandemia de COVID-19. Como el modelo se utiliza solamente para evaluar la metodología de inferencia no tenemos pretensión de dar valores equivalentes a los de estudios médicos sino que solamente tomamos valores razonables para nuestros objetivos. En \cite{Guan2020} se reporta un período de incubación de 4 días el cual es consistente con nuestra parametrización de la distribución Gamma ($\mu_E = k_E \theta_E = 4\text{ días}$). El tiempo medio de la etapa infecciosa elegido para casos leves es $\mu_{I_M} = k_{I_M} \theta_{I_M} = 8 \text{ días}$ y valores similares fueron utilizados en otros modelos \citep{Zhao2020, Ivorra2020}. Para el tiempo entre la enfermedad grave y a hospitalización usamos $\mu_{I_S} = k_{I_S} \theta_{I_S} = 8.1 \text{ días}$ lo cual está en el rango reportado en \cite{Faes2020}. Tomamos a la probabilidad de desarrollar sintomatología grave, $q_S$, como 10\% y a la probabilidad de que un paciente hospitalizado muera como 40\%. Esto resulta en una fatalidad total de 4\% que es alta pero en línea con los valores iniciales de la pandemia: en China, en Febrero del 2020 se registró un valor de 3.67\% \citep{Verity2020}. En Argentina se encontraron valores similares en los experimentos preliminares publicados en \cite{Evensen2020}. Este valor disminuyó sustancialmente pasada la primera etapa de la pandemia y aún más con el el comienzo de las vacunaciones masivas y el surgimiento de variantes menos letales. Para el valor de $\lambda$, que parametriza el valor medio de contactos diarios de cada agente, utilizamos distintas configuraciones en todos los experimentos por lo cual no proveemos un valor por defecto. La cantidad esperada de infecciones que un agente infectado produce en una población totalmente susceptible se puede computar como $\lambda \beta$. Con las elecciones que hicimos para $\lambda$ y los valores por defecto para $\beta_C$, $\beta_D$ y $q_C$ tenemos que esta cantidad es similar a la que se determina en los valores por defecto en el modelo en \cite{Kerr2020}. El vector de probabilidades para la distribución del tamaño de hogares por defecto lo tomamos como $p_H = (0.36\;\; 0.27\;\; 0.16\;\; 0.13\;\; 0.08)$. Esto implica que solo consideramos casas de hasta 5 habitantes. Estos valores son basados en la Encuesta Anual de Hogares 2019 para la Ciudad Autónoma de Buenos Aires.

\begin{table}[!ht]
    \centering
    \caption{\textbf{Parametrización por defecto para el epiABM.}}
    \label{table:default_params}
    \begin{tabular}{|c|c|c|}
      \hline
      Parámetro & Descripción & Valor \\ \hline
      $\beta_D$      & Probabilidad de infección en contactos domésticos               & $0.8$ \\ \hline
      $\beta_C$      & Probabilidad de infección en contactos casuales                 & $0.16$ \\ \hline
      $q_D$          & Probabilidad de muerte para hospitalizados          & $0.4$ \\ \hline
      $q_S$          & Probabilidad de que una infección sea grave              & $0.1$ \\ \hline
      $q_C$          & Probabilidad de que un contacto sea casual                   & $0.5$ \\ \hline
      $k_{E}$        & Parámetro de forma para la Gamma correspondiente a $E$   & $1.78$ \\ \hline
      $\theta_{E}$   & Parámetro de escala para la Gamma correspondiente a  $E$   & $2.25$ \\ \hline
      $k_{I_M}$      & Parámetro de forma para la Gamma correspondiente a  $I_M$ & $7.11$ \\ \hline
      $\theta_{I_M}$ & Parámetro de escala para la Gamma correspondiente a  $I_M$ & $1.13$ \\ \hline
      $k_{I_S}$      & Parámetro de forma para la Gamma correspondiente a  $I_S$ & $4.0$ \\ \hline
      $\theta_{I_S}$ & Parámetro de escala para la Gamma correspondiente a  $I_S$ & $1.0$ \\ \hline
      $k_{H}$        & Parámetro de forma para la Gamma correspondiente a  $H$   & $9.0$ \\ \hline
      $\theta_{H}$   & Parámetro de escala para la Gamma correspondiente a  $H$   & $0.9$ \\ \hline
    \end{tabular}
  \end{table}
