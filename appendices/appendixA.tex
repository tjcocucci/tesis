\chapter{Asimilación de datos}
\section{Algoritmo \textit{forward-backward}}\label{appendix:ffbs}

El algoritmo \textit{forward-backward} está especificado en \ref{algo:ffbs}. 

\begin{algorithm}[H]\label{algo:ffbs}
    % \SetAlgoLined
    \SetKwInOut{Input}{input}\SetKwInOut{Output}{output}

    \Input{
        \par
        Distribución inicial $p(\v x_0)$
        \par
        Distribución de transición $p(\v x_t | \v x_{t-1})$ para $t = 1, ..., T$
        \par
        Verosimilitud observacional $p(\v y_t | \v x_t)$ para $t = 1, ..., T$ 
    }
    \Output{
        \par
        Distribución predictiva $p(\v x_t | \v y_{1:t-1})$ para para $t = 1, ..., T$
        \par
        Distribución filtrante $p(\v x_t | \v y_{1:t})$ para $t = 1, ..., T$
        \par
        Distribución suavizante $p(\v x_t | \v y_{1:T})$ para $t = 1, ..., T$ 
    }

    \hrulefill

    \textit{Forward filter}\

    \For{$t=1, ..., T$}{
        Computar distribución predictiva:\
        $p(\v x_t | \v y_{1:t-1}) = \int p(\v x_t | \v x_{t-1}) p(\v x_{t-1} | \v y_{1:t-1}) d\v x_{t-1}$\

        Computar distribución filtrante:\
        $p(\v x_t | \v y_{1:t}) \propto p(\v y_t | \v x_t) p(\v x_t | \v y_{1:t-1})$\
    }

    \textit{Backward smoother}\

    \For{$t=T, ..., 1$}{
        Computar distribución suavizante:\
        $p(\v x_t | \v y_{1:T}) = 
        \int \frac{p(\v x_{t+1} | \v x_t)p(\v x_t |\v y_{1:t})}
            {p(\v x_{t+1} |\v y_{1:t})}
            p(\v x_{t+1} | \v y_{1:T}) d\v x_{t+1}$\
    }
    \caption{Algoritmo forward filter backward smoothing}
\end{algorithm}


La distribución predictiva se puede deducir integrando la distribución de transición pesando con la distribución filtrante del paso anterior:
\begin{align}
    p(\v x_t | \v y_{1:t-1}) &= \int p(\v x_t, \v x_{t-1} | \v y_{1:t-1}) d\v x_{t-1} && \text{Marginalización}\\
    &= \int p(\v x_t | \v x_{t-1}, \v y_{1:t-1}) p(\v x_{t-1} | \v y_{1:t-1}) d\v x_{t-1} && \text{Bayes}\\
    &=\int p(\v x_t | \v x_{t-1}) p(\v x_{t-1} | \v y_{1:t-1}) d\v x_{t-1} && \text{Propiedades HMM}
\end{align}

Por otro lado, para obtener la distribución filtrante podemos usar la distribución predictiva e incorporar la información de la observación $\v y_t$ de la siguiente manera:
\begin{align}
    p(\v x_t | \v y_{1:t}) &= \frac
            {p(\v y_t | \v x_t \v y_{1:t-1}) p(\v x_t | \v y_{1:t-1})}
            {p(\v y_t | \v y_{1:t-1})} && \text{Bayes}\\
    &= \frac{p(\v y_t | \v x_t) p(\v x_t | \v y_{1:t-1})}
            {p(\v y_t | \v y_{1:t-1})} && \text{Propiedades HMM} \label{eq:forward_backward_filter_complete}\\
    &\propto p(\v y_t | \v x_t) p(\v x_t | \v y_{1:t-1})
\end{align}

Para calcular la distribución suavizante necesitamos tener las distribuciones filtrantes y predictivas del forward-pass e iterar desde la última observación hasta la primera:
\begin{align}
    p(\v x_t | \v y_{1:T}) &= 
        \int p(\v x_t | \v x_{t+1}, \v y_{1:T})
             p(\v x_{t+1} | \v y_{1:T}) d\v x_{t+1}
        && \text{Marginalización} \\
    &= \int p(\v x_t | \v x_{t+1}, \v y_{1:t})
             p(\v x_{t+1} | \v y_{1:T}) d\v x_{t+1}
        && \text{Propiedades HMM} \\
    &= \int \frac{p(\v x_{t+1} | \v x_t, \v y_{1:t})p(\v x_t |\v y_{1:t})}
            {p(\v x_{t+1} |\v y_{1:t})}
            p(\v x_{t+1} | \v y_{1:T}) d\v x_{t+1}
        && \text{Bayes} \\
    &= \int \frac{p(\v x_{t+1} | \v x_t)p(\v x_t |\v y_{1:t})}
            {p(\v x_{t+1} |\v y_{1:t})}
            p(\v x_{t+1} | \v y_{1:T}) d\v x_{t+1}
        && \text{Propiedades HMM}
\end{align}

\section{Filtro de Kalman}\label{appendix:kf}

La fórmula \ref{eq:kf_mean_pred} para la media $\v x_t^f$ de la distribución predictiva puede ser deducida de la siguiente manera:

\begin{align*}
    \v x_t^f &= E[\v X_t | \v y_{1:t-1}] && \text{Definición de $\v x_t^f$} \\
    &= \int \v x_t p(\v x_t | y_{1:t-1}) d\v x_t && \text{Definición de $E$} \\
    &= \int \v x_t \int p(\v x_t | \v x_{t-1}) p(\v x_{t-1} | \v y_{1:t-1}) d\v x_{t-1} d\v x_t && \text{Ec. \ref{eq:forward_pred}} \\
    &= \int p(\v x_{t-1} | \v y_{1:t-1}) \int \v x_t p(\v x_t | \v x_{t-1}) d\v x_t d\v x_{t-1} && \text{Intercambio de $\int$}\\ 
    &= \int p(\v x_{t-1} | \v y_{1:t-1}) E[\v X_t | \v x_{t-1}] d\v x_{t-1} && \text{Definición de $E$}\\
    &= \int p(\v x_{t-1} | \v y_{1:t-1}) E[\v M_t \v x_{t-1} + \gv\eta_t] d\v x_{t-1} && \text{Modelo}\\
    &= \int p(\v x_{t-1} | \v y_{1:t-1}) \v M_t \v x_{t-1} d\v x_{t-1} && \text{$\v M_t$ lineal y $E[\gv\eta_t] = 0$}\\
    &= \v M_t \int p(\v x_{t-1} | \v y_{1:t-1}) \v x_{t-1} d\v x_{t-1} && \text{Modelo} \\
    &= \v M_t E[\v X_{t-1} | \v y_{1:t-1}] && \text{$\v M_t$ lineal}\\
    &= \v M_t \v x_{t-1}^a && \text{Definición de $\v x_t^a$} &&
\end{align*}

Por otro lado, la fórmula \ref{eq:kf_var_pred} para la matriz de covarianza $\v P_t^f$ de la distribución predictiva puede ser obtenida como se detalla a continuación:
\begin{align*}
    \v P_t^f &= Var[\v X_t | \v y_{1:t-1}] && \text{Definición de $\v P_t^f$}\\ 
    &= E[\v X_t \v X_t^T | \v y_{1:t-1}] - E[\v X_t | \v y_{1:t-1}] E[\v X_t | \v y_{1:t-1}]^T && \text{$Var[\v X] = E[\v X \v X^T] - E[\v X]E[\v X]^T$} \\
    &= E[\v X_t \v X_t^T | \v y_{1:t-1}] - \v M_t \v x_{t-1}^a \v x_{t-1}^{aT} \v M_t^T && \text{Ec. \ref{eq:kf_mean_pred}}
\end{align*}
Ahora desarrollamos el valor esperado del primer término:
\begin{align*}
    E[&\v X_t \v X_t^T | \v y_{1:t-1}] = \int \v x_t \v x_t^T p(\v x_t | \v y_{1:t-1}) d\v x_t && \\
     &= \int \v x_t \v x_t^T \int p(\v x_t | \v x_{t-1}) p(\v x_{t-1} | \v y_{1:t-1}) d\v x_{t-1} d\v x_t && \text{Ec. \ref{eq:forward_pred}}\\
    &= \int p(\v x_{t-1} | \v y_{1:t-1}) \int \v x_t \v x_t^T p(\v x_t | \v x_{t-1}) d\v x_t d\v x_{t-1} && \text{Intercambio de $\int$}\\
    &= \int p(\v x_{t-1} | \v y_{1:t-1}) E[\v X_t \v X_t^T | \v x_{t-1}] d\v x_{t-1} && \text{Definición de $E$}\\
    &= \int p(\v x_{t-1} | \v y_{1:t-1}) (Var[\v X_t | \v x_{t-1}] && \text{$E[\v X \v X^T] = Var[\v X]^T + E[\v X]E[\v X]^T$}\\
    &\hspace{2em} + E[\v X_t | \v x_{t-1}]E[\v X_t | \v x_{t-1}]^T) d\v x_{t-1} && \\
    &= \int p(\v x_{t-1} | \v y_{1:t-1}) Var[\v X_t | \v x_{t-1}] d \v x_{t-1} && \text{Linealidad de $\int$}\\
    &\hspace{2em} + \int p(\v x_{t-1} | \v y_{1:t-1}) E[\v X_t | \v x_{t-1}]E[\v X_t | \v x_{t-1}]^T) d\v x_{t-1} && \\
    &= \int p(\v x_{t-1} | \v y_{1:t-1}) \v Q_t d \v x_{t-1} && \text{Ec. \ref{eq:kf_forward}}\\
    &\hspace{2em} + \int p(\v x_{t-1} | \v y_{1:t-1}) \v M_t \v x_{t-1} \v x_{t-1}^T \v M_t^T d\v x_{t-1} \\
    &= \v Q_t + \v M_t \int p(\v x_{t-1} | \v y_{1:t-1}) \v x_{t-1} \v x_{t-1}^T d\v x_{t-1} \v M_t^T && \text{$\v M_t$ lineal}\\
    &= \v Q_t + \v M_t E[\v X_{t-1} \v X_{t-1}^T | \v y_{1:t-1}] \v M_t^T && \text{Definición de $E$}\\
    &= \v Q_t + \v M_t E[\v X_{t-1} | \v y_{1:t-1}]E[\v X_{t-1} | \v y_{1:t-1}]^T \v M_t^T && \text{$E[\v X \v X^T] = Var[\v X] + E[\v X]E[\v X]^T$} \\
    &\hspace{2em} + \v M_t Var[\v X_{t-1} | \v y_{1:t-1}] \v M_t^T \\
    &= \v Q_t + \v M_t \v x_{t-1}^a \v x_{t-1}^{aT} \v M_t^T + \v M_t \v P_{t-1}^a \v M_t^T && \text{Definición de $\v x_{t-1}^a$ y $\v P_{t-1}^a$}\\
\end{align*}
Por lo tanto al combinar las expresiones obtenemos el resultado:
\begin{align*}
    \v P_t^f &= \v Q_t + \v M_t \v P_{t-1}^a \v M_t^T
\end{align*}

Para obtener las fórmulas de la media y covarianza de la distribución filtrante debemos usar la ecuación de análisis del algoritmo \textit{forward-backwards}:
\begin{align*}
    p(\v x_t | \v y_{1:t}) &\propto p(\v y_t | \v x_t) p(\v x_t | \v y_{1:t-1}) && \text{Ec. \ref{eq:forward_filt}} \\
    &\propto \exp((\v y_t - \v H_t \v x_t)^T \v R_t^{-1} (\v y_t - \v H_t \v x_t) && \text{Densidades Gaussianas}\\ 
    &+ (\v x_t - \v x_t^f)^T \v (\v P_t^{f})^{-1} (\v x_t - \v x_t^f)) \\
    &\propto \exp(\v x_t^T (\v (\v P_t^f)^{-1} + \v H^T \v R_t^{-1} \v H_t) \v x_t && \text{Distribución}\\
    &- 2 \v x_t^T (\v H^T \v R_t^{-1} \v y_t + (\v P_t^{f})^{-1} \v x_t)) \\
    &= \exp(\v x_t^T \v A \v x_t - 2\v x_t^T \v v) && \text{Renombre}\\
    &= \exp((\v x_t - \v A^{-1}\v v)^T \v A (\v x_t - \v A^{-1}\v v) - \v v^T \v A \v v) && \text{Completar cuadrados}\\
    &\propto \exp((\v x_t - \v A^{-1}\v v)^T \v A (\v x_t - \v A^{-1}\v v))
\end{align*}
donde hemos utilizado la siguiente nomenclatura:
\begin{align*}
    \v A &= \v (\v P_t^f)^{-1} + \v H^T \v R_t^{-1} \v H_t \\
    \v v &= \v H^T \v R_t^{-1} \v y_t + (\v P_t^{f})^{-1} \v x_t
\end{align*}.

La expresión que obtuvimos implica que la distribución filtrante es Gaussiana con media $\v A^{-1}\v v$ y covarianza $\v A^{-1}$. Vamos a desarrollar estas expresiones para obtener la formulación clásica del filtro de Kalman. Para ello, necesitaremos usar la siguiente identidad matricial de Woodbury \citep{Golub1996}:
\begin{align*}
    (\v A + \v C \v B \v C^T)^{-1} = \v A^{-1} - \v A^{-1} \v C (\v B^{-1} + \v C^T \v A^{-1} \v C)^{-1} \v C^T \v A^{-1}
\end{align*}

Tenemos entonces que:
\begin{align*}
    \v P_t^a &= \v A^{-1} && \\
    &= ((\v P_t^f)^{-1} + \v H^T \v R_t^{-1} \v H_t)^{-1} && \\
    &= \v P_t^f - \v P_t^f \v H_t^T (\v R_t + \v H \v P_t^f \v H^T)^{-1} \v H_t \v P_t^f && \text{Identidad de Woodbury}\\
    &= \v P_t^f - \v K_t \v H_t \v P_t^f && \\
    &= (\v I - \v K_t \v H_t) \v P_t^f && \\
\end{align*}
donde hemos definido a $\v K_t = \v P_t^f \v H_t^T (\v R_t + \v H \v P_t^f \v H^T)^{-1}$. Esta matriz es denominada matriz de ganancia de Kalman. Para desarrollar la expresión de la media de la distribución además usaremos la notación $\v S_t = (\v R_t + \v H \v P_t^f \v H^T)^{-1}$, con la cual la ganancia de Kalman se puede expresar como $\v K_t = \v P_t^f \v H_t^T \v S_t$ y podemos obtener la fórmula para $\v x_t^a$ de la siguiente manera:
\begin{align*}
    \v x_t^a &= \v A^{-1} \v v && \\
    &= (\v I - \v K_t \v H_t) (\v P_t^f \v H^T \v R_t^{-1} \v y_t + (\v P_t^{f})^{-1} \v x_t) && \\
    &= \v x_t - \v K_t \v H_t \v x_t + \v P_t \v H_t^T \v R_t^{-1} \v y_t - \v K_t \v H_t \v P_t \v H_t^T \v R_t^{-1} \v y_t && \\
    &= \v x_t - \v K_t \v H_t \v x_t + \v P_t \v H_t^T \v R_t^{-1} \v y_t - \v K_t \v H_t \v P_t \v H_t^T \v R_t^{-1} \v y_t && \\
    &= \v x_t - \v K_t \v H_t \v x_t + \v P_t \v H_t^T \v S_t \v S_t^{-1} \v R_t^{-1} \v y_t - \v K_t \v H_t \v P_t \v H_t^T \v R_t^{-1} \v y_t && \\
    &= \v x_t - \v K_t \v H_t \v x_t + \v K_t (\v R_t + \v H \v P_t^f \v H^T)^{-1} \v R_t^{-1} \v y_t - \v K_t \v H_t \v P_t \v H_t^T \v R_t^{-1} \v y_t && \\
    &= \v x_t - \v K_t \v H_t \v x_t + \v K_t \v y_t && \\
    &= \v x_t + \v K_t (\v y_t - \v H_t \v x_t) && \\
\end{align*}

\section{Filtro de partículas}\label{appendix:pf}

Hacemos aquí una deducción de el filtro de partículas SIR. El objetivo es obtener representaciones de partículas $\{\v x_t^{(i)}, w_t^{(i)}\}_{i=1}^{N}$ tal que sea una aproximación empírica de $p(\v x_t | \v y_{1:t})$. Vamos a comenzar por considerar la distribución conjunta de las variables de estado condicionadas a las observaciones, $p(\v x_{0:t} | \v y_{1:t})$. Estamos considerando a las variables de estado desde el tiempo $0$ hasta el $t$, lo cual significa que esta es la densidad de probabilidad de una trayectoria de las variables de estado en el tiempo. Claramente, si tenemos una muestra de esta distribución, las componentes correspondientes al tiempo $t$ constituirán una muestra de la probabilidad filtrante marginalizada $p(\v x_t | \v y_{1:t})$. Utilizando las propiedades Markovianas y la independencia condicional de las observaciones del modelo de Markov escondido podemos escribir:
\begin{align*}
    p(\v x_{0:t} | \v y_{1:t}) \propto p(\v x_{0:t-1} | \v y_{1:t}) p(\v x_t | \v x_{t-1}) p(\v y_t | \v x_t)
\end{align*}

Si muestreamos trayectorias $\{\v x_{0:t}^{(i)}\}_{i=1}^N$ de una probabilidad propuesta $q$ vamos a obtener que los pesos de importancia son 
\begin{align}\label{eq:importance_weights_general}
    w_t^{(i)} \propto \frac{p(\v x_{0:t-1} | \v y_{1:t}) p(\v x_t | \v x_{t-1}) p(\v y_t | \v x_t)}{q(\v x_{0:t})}
\end{align}

Adicionalmente consideraremos que $q$ cumple 
\begin{align}\label{eq:proposal_factor}
    q(\v x_{0:t} | \v y_{1:t}) = q(\v x_t | \v x_{0:t-1}, \v y_{1:t}) q(\v x_{0:t-1} | \v y_{1:t-1})
\end{align}
Esta factorización implica que si tenemos una muestra de la trayectoria $\v x_{0:t-1}^{(i)} \sim q(\v x_{0:t-1} | \v y_{1:t-1})$ entonces se puede obtener una muestra de la trayectoria hasta el tiempo $t$ incorporando la última componente muestreada como $\v x_t^{(i)} \sim q(\v x_t | \v x_{0:t-1}, \v y_{1:t})$.

Si entonces introducimos \ref{eq:proposal_factor} en \ref{eq:importance_weights_general}, tenemos que los pesos de importancia pueden ser computados como:
\begin{align*}
    w_t^{(i)} &\propto \frac{p(\v x_{0:t-1} | \v y_{1:t}) p(\v x_t | \v x_{t-1}) p(\v y_t | \v x_t)}{q(\v x_{0:t-1} | \v y_{1:t-1}) q(\v x_t | \v x_{0:t-1},  \v y_{1:t})} \\
    &\propto w_{t-1}^{(i)} \frac{p(\v x_t | \v x_{t-1}) p(\v y_t | \v x_t)}{q(\v x_t | \v x_{0:t-1},  \v y_{1:t})}
\end{align*}

Si adicionalmente dotamos a $q$ de ``Markovianidad'' en el sentido que $q(\v x_t | \v x_{0:t-1} \v y_{1:t}) = q(\v x_t | \v x_{t-1} \v y_t)$, entonces los pesos solamente dependen de $\v x_t$ y no de toda la trayectoria $\v x_{0:t-1}$. De esta manera se puede hacer filtrado de manera secuencial. Con estas suposiciones obtenemos la forma general de los pesos de el filtro de partículas SIR:
\begin{align*}
    w_t^{(i)} &\propto w_{t-1}^{(i)} \frac{p(\v x_t | \v x_{t-1}) p(\v y_t | \v x_t)}{q(\v x_t | \v x_{t-1}, \v y_t)}
\end{align*}

\section{EnKF} \label{appendix:enkf}

Para mostrar que la formulación del EnKF estocástico es correcta, basta probar que la media y covarianza que se obtiene de considerar a los miembros del ensamble como variables aleatorias coinciden con las del filtro de Kalman tradicional, el cual da una solución exacta al problema. Supongamos inductivamente que las partículas a tiempo $t-1$ cumplen con esto y veamos que podemos obtener las fórmulas correctas a tiempo $t$.

Las partículas del pronóstico están definidas como 
$$ \v x_t^{f, (i)} = \v M_t \v x_{t-1}^{a, (i)} + \gv\eta_t^{(i)}$$
con $\gv\eta_t^{(i)} \sim \mathcal{N}(\v 0, \v Q_t)$. Por lo tanto si calculamos la media y covarianza, obtenemos las mismas fórmulas que para el filtro de Kalman tradicional \ref{eq:kf_mean_pred} y \ref{eq:kf_var_pred}:
\begin{align*}
    E[\v x_t^{f, (i)}] &= E[\v M_t \v x_{t-1}^{a, (i)} + \gv\eta_t^{(i)}] \\
    &= \v M_t E[\v x_{t-1}^{a, (i)}] + E[\gv\eta_t^{(i)}] \\
    &= \v M_t \v x_{t-1}^a \\
    Var[\v x_t^{f, (i)}] &= Var[\v M_t \v x_{t-1}^{a, (i)} + \gv\eta_t^{(i)}] \\
    &= \v M_t Var[\v x_{t-1}^{a, (i)}] \v M_t^T + Var[\gv\eta_t^{(i)}] \\
    &= \v M_t \v P_t^a \v M_t^T + \v Q_t 
\end{align*}
donde hemos usado la independencia de $\gv\eta_t^{(i)}$ y $\v x_{t-1}^{a, (i)}$ y que $Var[\v A \v X] = \v A Var[\v X] \v A^T$ cuando $\v A$ es un operador lineal. Las partículas del pronóstico entonces constituyen efectivamente una muestra de la distribución del pronóstico.

Las partículas del análisis en el EnKF estocástico están definidas por 
$$ \v x_t^{a, (i)} = \v x_t^{f, (i)} + \widehat{\v K_t} (\v y_t - (\v H_t \v x_t^{f, (i)} + \v v_t^{(i)}))$$
donde $\v v_t^{(i)} \sim \mathcal{N}(\v 0, \v R_t)$ y la aproximación a la ganancia de Kalman es $\widehat{\v K}_t = \widehat{\v P}_t^f \v H_t^T(\v R_t + \v H_t \widehat{\v P}_t^f \v H_t^T)^{-1}$ con $\widehat{\v P}_t^f$ denotando la covarianza muestral del ensamble de pronóstico. Probaremos que en el caso ideal en que estemos usando la matriz de ganancia verdadera (es decir la que utiliza la covarianza exacta del pronóstico, $\v P_t^f$) la media y covarianza del ensamble del análisis coinciden con las del filtro de Kalman \ref{eq:kf_mean_filter} y \ref{eq:kf_var_filter}.
\begin{align*}
    E[\v x_t^{a, (i)}] &= E[\v x_t^{f, (i)} + \v K_t (\v y_t - (\v H_t \v x_t^{f, (i)} + \v v_t^{(i)}))] \\
    &= \v x_t^f + \v K_t E[(\v y_t - (\v H_t \v x_t^{f, (i)} + \v v_t^{(i)}))] \\
    &= \v x_t^f + \v K_t (\v y_t - E[\v H_t \v x_t^{f, (i)}]) \\
    &= \v x_t^f + \v K_t (\v y_t - \v H_t \v x_t^f)
\end{align*}
Por otro lado, para la varianza del ensamble de análisis tenemos que:
\begin{align*}
    Var[\v x_t^{a, (i)}] &= Var[\v x_t^{f, (i)} + \v K_t (\v y_t - (\v H_t \v x_t^{f, (i)} + \v v_t^{(i)}))] \\
    &= Var[\v x_t^{f, (i)}] + Var[\v K_t (\v y_t - (\v H_t \v x_t^{f, (i)} + \v v_t^{(i)}))] \\
    &+ 2 Cov[\v x_t^{f, (i)}, \v K_t (\v y_t - (\v H_t \v x_t^{f, (i)} + \v v_t^{(i)}))] \\
    &= \v P_t^f + \v K_t Var[(\v H_t \v x_t^{f, (i)} + \v v_t^{(i)})]\v K_t^T + 2 Cov[\v x_t^{f, (i)}, - \v K_t \v H_t \v x_t^{f, (i)}] \\ 
    &= \v P_t^f + \v K_t (\v H_t \v P_t^f \v H_t^T + \v R_t)\v K_t^T - 2 \v K_t \v H_t \v P_t^f \\
    &= \v P_t^f + \v P_t^f \v H_t^T (\v H_t \v P_t^f \v H_t^T + \v R_t)^{-1}(\v H_t \v P_t^f \v H_t^T + \v R_t)\v K_t^T - 2 \v K_t \v H_t \v P_t^f \\
    &= \v P_t^f + \v P_t^f \v H_t^T \v K_t^T - 2 \v K_t \v H_t \v P_t^f \\
    &= \v P_t^f + (\v K_t \v H_t \v P_t^f)^T - 2 \v K_t \v H_t \v P_t^f \\
    &= \v P_t^f - \v K_t \v H_t \v P_t^f \\
    &= (\v I - \v K_t \v H_t) \v P_t^f 
\end{align*}
donde hemos usado la propiedad de la varianza de una suma ($Var[\v X + \v Y] = Var[\v X] + Var[\v Y] + Cov[\v X, \v Y]$), la independencia de $\v x_t^{f, (i)}$ con $\v v_t^{(i)}$ y el hecho que $\v K_t \v H_t \v P_t^f$ es una matriz simétrica.
