\chapter{Algoritmo \textit{forward-backward}}\label{appendix:ffbs}

El algoritmo forward-backward está especificado en \ref{algo:ffbs}. 

\begin{algorithm}[H]\label{algo:ffbs}
    % \SetAlgoLined
    \SetKwInOut{Input}{input}\SetKwInOut{Output}{output}

    \Input{
        \par
        Distribución inicial $p(\v x_0)$
        \par
        Distribución de transición $p(\v x_t | \v x_{t-1})$ para $t = 1, ..., T$
        \par
        Verosimilitud observacional $p(\v y_t | \v x_t)$ para $t = 1, ..., T$ 
    }
    \Output{
        \par
        Distribución predictiva $p(\v x_t | \v y_{1:t-1})$ para para $t = 1, ..., T$
        \par
        Distribución filtrante $p(\v x_t | \v y_{1:t})$ para $t = 1, ..., T$
        \par
        Distribución suavizante $p(\v x_t | \v y_{1:T})$ para $t = 1, ..., T$ 
    }

    \hrulefill

    \textit{Forward filter}\

    \For{$t=1, ..., T$}{
        Computar distribución predictiva:\
        $p(\v x_t | \v y_{1:t-1}) = \int p(\v x_t | \v x_{t-1}) p(\v x_{t-1} | \v y_{1:t-1}) d\v x_{t-1}$\

        Computar distribución filtrante:\
        $p(\v x_t | \v y_{1:t}) \propto p(\v y_t | \v x_t) p(\v x_t | \v y_{1:t-1})$\
    }

    \textit{Backward smoother}\

    \For{$t=T, ..., 1$}{
        Computar distribución suavizante:\
        $p(\v x_t | \v y_{1:T}) = 
        \int \frac{p(\v x_{t+1} | \v x_t)p(\v x_t |\v y_{1:t})}
            {p(\v x_{t+1} |\v y_{1:t})}
            p(\v x_{t+1} | \v y_{1:T}) d\v x_{t+1}$\
    }
    \caption{Algoritmo forward filter backward smoothing}
\end{algorithm}


La distribución predictiva se puede deducir integrando la ditribución de transición pesando con la distribución filtrante del paso anterior:
\begin{align}
    p(\v x_t | \v y_{1:t-1}) &= \int p(\v x_t, \v x_{t-1} | \v y_{1:t-1}) d\v x_{t-1} && \text{Marginalización}\\
    &= \int p(\v x_t | \v x_{t-1}, \v y_{1:t-1}) p(\v x_{t-1} | \v y_{1:t-1}) d\v x_{t-1} && \text{Bayes}\\
    &=\int p(\v x_t | \v x_{t-1}) p(\v x_{t-1} | \v y_{1:t-1}) d\v x_{t-1} && \text{Propiedades HMM}
\end{align}

Por otro lado, para obtener la distribución filtrante podemos usar la distribución predictiva e incorporar la información de la observación $\v y_t$ de la siguiente manera:
\begin{align}
    p(\v x_t | \v y_{1:t}) &= \frac
            {p(\v y_t | \v x_t \v y_{1:t-1}) p(\v x_t | \v y_{1:t-1})}
            {p(\v y_t | \v y_{1:t-1})} && \text{Bayes}\\
    &= \frac{p(\v y_t | \v x_t) p(\v x_t | \v y_{1:t-1})}
            {p(\v y_t | \v y_{1:t-1})} && \text{Propiedades HMM}\\
    &\propto p(\v y_t | \v x_t) p(\v x_t | \v y_{1:t-1})
\end{align}

Para calcular la distribución suavizante necesitamos tener las distribuciones filtrantes y predictivas del forward-pass e iterar desde la última observación hasta la primera:
\begin{align}
    p(\v x_t | \v y_{1:T}) &= 
        \int p(\v x_t | \v x_{t+1}, \v y_{1:T})
             p(\v x_{t+1} | \v y_{1:T}) d\v x_{t+1}
        && \text{Marginalización} \\
    &= \int p(\v x_t | \v x_{t+1}, \v y_{1:t})
             p(\v x_{t+1} | \v y_{1:T}) d\v x_{t+1}
        && \text{Propiedades HMM} \\
    &= \int \frac{p(\v x_{t+1} | \v x_t, \v y_{1:t})p(\v x_t |\v y_{1:t})}
            {p(\v x_{t+1} |\v y_{1:t})}
            p(\v x_{t+1} | \v y_{1:T}) d\v x_{t+1}
        && \text{Bayes} \\
    &= \int \frac{p(\v x_{t+1} | \v x_t)p(\v x_t |\v y_{1:t})}
            {p(\v x_{t+1} |\v y_{1:t})}
            p(\v x_{t+1} | \v y_{1:T}) d\v x_{t+1}
        && \text{Propiedades HMM}
\end{align}
