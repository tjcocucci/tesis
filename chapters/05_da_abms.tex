\chapter{Asimilación de datos en modelos basados en agentes}

\section{Modelos basados en agentes}

A diferencia de los modelos de predeicción con ecuaciones diferenciales, los modelos basados en agentes (conocidos como ABMs por sus siglas en inglés) se basan en un paradigma diferente. Estos modelan de manera explícita las características y comportamiento de individuos o agentes que interactúan. El comportamiento conjunto de la población de agentes se utiliza como modelo de un sistema complejo \citep{Bonabeau2002}. Incluso reglas de interacción simples pueden resultar en que los agentes se organicen de manera autónoma y que el comportamiento colectivo de la población emerja de manera \textit{bottom-up} desde la escala micro del modelado de los individuos \citep{Helbing2012}. En general, los ABMs requieren una gran capacidad computacional ya que las poblaciones de agentes pueden ser muy grandes. Actualmente es factible utilizarlos de manera operacional y existen ejemplos de su aplicación en epidemiología, ecología, economía y sociología \citep{Vynnycky2010, Grimm2005, Tesfatsion2006, Epstein1996}.

Los modelos de agentes tienen dos componentes que los describen: 
\begin{enumerate}
    \item Las descripciones de los agentes
    \item Las reglas de comportamiento e interacción
\end{enumerate}
Los agentes pueden ser vistos como un tipo de dato con diferentes campos de manera que cada individuo se define por el valor de estos atributos. Para completar al modelo se agrega el comportamiento e interacción de los agentes, en general mediante reglas que pueden contener componentes estocásticos. Las acciones que realicen los individuos baje estas disposiciones provocarán cambios en los valores de sus atributos. En principio estos atributos pueden ser cualquier tipo de dato, de manera que los agentes pueden ser programados como tipos de datos compuestos (como estructuras en C) o incluso a través de clases en lenguajes orientados a objetos. Por ejemplo, se pueden usar números reales para representar coordenadas espaciales, variables categóricas para clases sociales, números enteros para edades, etc.

Para un modelar la dispersión de una enfermedad es apropiado construir ABMs epidemiológicos tomando una población de agentes en el que cada uno de ellos representa a una persona (o individuo susceptible a contraer y/o transmitir la enfermedad) \citep{Roche2011}. Utilizando variables categóricas se puede etiquetar a cada agente con su status o categoría epidemiológica (susceptible, infectado, etc.) y estas etiquetas pueden ser cambiadas cuando existe un contacto infeccioso o a medida que la enfermedad se desarolla en el tiempo. El modelado de los contactos entre agentes admite múltiples representaciones, se pueden hacer mezclas al azar, a través de grupos, con redes de contactos o modelando explícitamente la movilidad de los agentes en el espacio. El uso de ABMs para epidemiología tiene la virtud de que permite modelar de una manera explícita, y con gran detalle, características relevantes del sistema que suceden en la micro-escala de la intereacción entre individuos. Por ejemplo, la disminución de la inmunidad puede ser representada en cada individuo a través de un atributo. Es posible modelar de manera natural medidas de control como cuarentenas, reastreo de contactos o efectos de vacunación \cite{Silva2020}. Muchos modelos de ecuaciones diferenciales asumen que dentro de cada compartimento la mezcla entre individuos es homogénea; para ABMs epidemiológicos, al disponer de caracterizaciones de cada individuo, es posible obtener patrones de interacción mas ricos a través de redes de contacto o el mecanismo que resulte más conveniente. De esta manera se pueden capturar efectos difíciles de representar con modelos de ecuaciones diferenciales. A pesar de disponer de información de cada agente, en la simulación con ABMs el interés está en el comportamiento global del sistema. Se suele contar con una función que de alguna manera resuma al estado de la población como un todo a través de un conjunto de variables que llamaremos variables agregadas. En general, si tenemos que la población de agentes es $A$, llamaremos $\v x$ a las variables agregadas y $\phi$ a la función sintetizante que realiza el mapeo, de manera que:
\begin{align}
    \phi (A) = \v x
\end{align}
El comportamiento de las variables agregadas emergerá del comportamiento a nivel individual del ABM. En el caso de modelos epidemiológicos es razonable, por ejemplo, utilizar como resumen el conteo de la cantidad de individuos en cada categoría para saber cuantos susceptibles, infectados, recuperados, etc. hay.

Con el surgimiento de la pandemia de COVID-19 se han desarrollado una gran diversidad de ABMs. Algunos de estos incorporan, entre otras características, estructura de edades y redes de contactos a través de escuelas, casas, lugares de trabajo que permiten modelar de manera más realista las mezclas que dan lugar a los contacios \citep{Kerr2020,Flaxman2020,Simoy2021}. Además de que el aumento de la capacidad de cómputo hace más viable la utilización de ABMs, el gran caudal de datos recolectados a nivel individual a través de dispositivos electrónicos es otro gran aliciente para el incremento en su popularidad. En \cite{Aleta2020} se utilizan datos demográficos y de movilidad para construir las redes de contactos y distribución de hogares en un ABM que permite evaluar los efectos de las intervenciones no-farmacéuticas.

A pesar de la flexibilidad y expresividad que permite el modelado mediante agentes, sigue siendo necesario calibrarlos adecuadamente. En general, el aumento en complejidad viene acompañado de un aumento en el número de parámetros, los cuales no siempre son especificables a través del conocimento que se tenga sobre el sistema. Ha habido avances en el desarrollo de técnicas de inferencia para estimar parámetros en ABMs utilizando observaciones sobretodo orientados a la maximización de la verosimilitud. En particular en \cite{Hooten2020} se discuten una variedad de métodos para maximizar aproximaciones de la verosimilitud usando cómputo bayesiano aproximado (o ABM por \textit{Approximate Bayesian Computation}) junto con técnicas de MCMC (\textit{Markov Chain Monte Carlo}).

Como los ABMs se enmarcan en un contexto de sistemas con evolución temporal parcialmente observados, es posible pensar que la caja de herramientas provista por la asimilación de datos secuencial tiene potencial para realizar las tareas de inferencia. Un trabajo pionero en esta dirección es \cite{Ward2016} en el que se utiliza el EnKF para asimilar datos de cámaras de pisadas en calles Leeds con un modelo de agentes para estudiar el comportamiento del tráfico peatonal. A pesar de obtener resultados satisfactorios, se señala la dificultad asociada a la sensiblidad respecto a parámetros del modelo y la necesidad optimizar el código para modelos de gran tamaño. Como mencionamos anteriormente, el interés no necesariamente está puesto en estado de cada individuo sino en el estado global del sistema o deun conjunto de variables que den una descripción de la población de agentes que se considere relevante. Además, a pesar de que el estado a la escala microscópica determina de manera total al sistema y que el objetivo de la inferencia fuera, idealmente, representar esta escala de la manera más precisa posible, sucede que usualmente las observaciones que se tienen del sistema son de las varables sintetizantes, es decir de la escala macroscópica. Es decir, las observaciones pueden no tener suficiente ``resolución'' como para que se pueda determinar el estado de los atributos de cada agente. Nuestra propuesta, publicada en \cite{Cocucci2022}, es la de asimilar datos sobre las variables agregadas mediante técnicas basadas en ensambles. El mapeo de la escala microscópica a macroscópica está dado por la función sintetizante $\phi$ pero no tenemos una representación para su inversa. De hecho $\phi$ puede no ser inyectiva y por lo tanto no-invertible: muchas configuraciones de los agentes pueden resultar en el mismo estado de las variables agregadas. Esta situación debe considerarse para hacer un emparejamiento entre los estados macroscópicos observados y el estado de los atributos de los individuos. En este capítulo presentamos un esquema general de aplicación de métodos de asimilación de datos por ensambles para ABMs y dos implementaciones particulares para un ABM epidemiológico que también especificamos a continuación.

\section{Modelo epiABM}

Aquí especificamos un ABM epidemiológico diseñado para modelar la dinámica de contagio de enfermedades infecciosas, en particular para el COVID-19 y que denominaremos epiABM. Este es un modelo sencillo para evaluar el rendimiento de la metodología para aplicar asimilación de datos por ensambles en ABMs y no tiene el propósito de ser un modelo para realizar predicciones o asesorar en toma de decisiones.

El estado de salud de cada agente está descripto por una etiqueta para representar una entre siete categorías. Tenemos a los susceptibles a contraer la enfermedad ($S$), los expuestos a la enfermedad pero que aún no son infecciosos debido a que el virus está incubando ($E$). Después de considerarse expuestos pasan a una de dos posibles clases de infecciosos. Los infectados leves ($I_M$) son los que desarrollan una sintomatología que no requiere hospitalización y que eventualmente se recuperan. Los infectados graves ($I_S$) son los que requerirán hospitalización. Por su parte, los hospitalizados pueden recuperarse o morir. La categoría $R$ denomina a los recuperados y la $D$ a los muertos. Asumimos que los recuperados adquieren inmunidad permanente, lo cual no es una suposición realista para simulaciones a largo plaze de COVID-19 puesto que las reinfecciones son posibles. El diagrama de la figura \ref{dia:epi_abm} representa de manera esquemática la progresión de la enfermedad a través de estos estadíos.

\begin{figure}
    \begin{center}
        \tikzstyle{block} = [rectangle, draw, fill=blue!20, 
        text width=4em, text centered, rounded corners, minimum height=3em]
        \tikzset{line/.style={draw, very thick, color=black!100, -latex'}}
        \centering
        \begin{tikzpicture}[node distance = 2cm, auto]
            \tikzstyle{block} = [rectangle, draw, fill=blue!20, 
            text width=2em, text centered, rounded corners, minimum height=3em]
            
            % Place nodes
            \node [block] (S) {$S$};
            \node [block, right of=S] (E) {$E$};
            \node [block, above right = 0.05cm and 1cm of E] (IM) {$I_M$};
            \node [block, below right = 0.05cm and 1cm of E] (IS) {$I_S$};
            \node [block, right of=IS] (H) {$H$};
            \node [block, right of=H] (D) {$D$};
            \node [block] (R) at (IM-|D) {$R$};
            
            % Draw edges
            \draw[line] (S) -- (E);
            \draw[line] (E) |- node [below] {$q_S$} (IS);
            \draw[line] (E) |- node [above] {$1-q_S$} (IM);
            \draw[line] (IM) -- (R);
            \draw[line] (IS) -- (H);
            \draw[line] (H) -- node [above] {$q_D$} (D);
            \draw[line] (H) -- node [left = 0.01cm] {$1-q_D$} (R);
        \end{tikzpicture}
    \end{center}
    \caption{Diagrama de las categorías epidemiológicas dee modelo epiABM.}
    \label{dia:epi_abm}
\end{figure}

En cada paso temporal, que consideraremos de un día, cada agente tendrá un número de contactos muestreado de una distribución Poisson de parámetro $\lambda$. Los agentes susceptibles podrán resultar infectados por un contacto con un agente infeccioso. El tiempo de permanencia de un agente en las categorías ``intermedias'' ($E$, $I_M$, $I_S$, $H$) está muestreado de una distribución Gamma. Si nombramos a este tiempo $\tau_c$ con $c \in \{E, I_S, I_M, H\}$ entonces $\tau_c \sim \Gamma(k_c, \theta_c)$ donde $k_c$ y $\theta_c$ son los parámetros de forma y escala respectivamente. Con esta parametrización, la media $\mu_c$ y la varianza $\sigma^2_c$ cumplen con las relaciones $\mu_c = k_c \theta_c$ y $\sigma^2_c = k_c \theta_c^2$. El tiempo $\tau_c$ es muestreado para cada agente cuando este entra a la categoría $c$ y cuando este tiempo expira, el agente pasa a la siguiente categoría. Cuando un agente sale de la categoría $E$, tiene una probabilidad $q_S$ de tener una enfermedad severa y $q_M = 1 - q_S$ de que su enfermedad sea leve. Análogamente, los hospitalizados tienen una probabilidad $q_D$ de morir y $q_R = 1 - q_D$ de recuperarse.

Además de la estructura dada por el estatus epdemiológico, incluímos información demográfica y geográfca. El modelo considera una ciudad dividida en $N_{loc}$ comunas. Cada agente vive en una casa (solo o con otros agentes) en una determinada comuna. Los contactos entre agentes pueden entonces ser definidos en base a esta estructura sencilla. Diferenciaremos entre contactos domésticos y casuales. Cada uno de los contactos diarios de los agentes tiene una probablilidad $q_C$ de ser casual y $1 - q_C$ de ser doméstico. Los contactos domésticos se dan solamente entre miembros de un mismo hogar mientras que los casuales pueden ser, potencialmente con cualquier otro agente. Llamamos $\beta_D$ (respectivamente $\beta_C$) a la probabilidad de que un contacto doméstico (respectivamente casual) sea infeccioso. De esta manera podemos escribir una expresión para el valor esperado para la probabilidad de infección global como $\beta = q_C \beta_C + (1 - q_C) \beta_D$. Por su parte, los contactos casuales van a estar mediados por la conectividad entre los diferentes barrios. Llamaremos $C_{ij}$ a la probabilidad de que un contacto casual de un agente del barrio $j$ se de con un agente del barrio $i$. Esto resulta en una matriz $C$ de tamaño $N_{loc} \times N_{loc}$ que nombraremos matriz de contacto. Los elementos de la diagonal corresponden a la probabilidad de que un contacto casual se de entre habitantes del mismo barrio mientras que los elementos fuera de ella tienen que ver con los contactos entre habitantes de barrios distintos. La matriz $C$ codifica entonces la movilidad entre barrios que a su vez se relaciona a las actividades sociales y laborales de la ciudad. El diseño de esta matriz puede incluír diferentes características de la estructura social y geográfica de la ciudad. Por ejemplo, sería esperable que los agentes transiten con más frecuencia su propio barrio, lo que implicaría valores más altos en los elementos de la diagonal y más pequeños por fuera de esta. Por otro lado, en caso de tener un barrio con más tránsito, por ejemplo un barrio céntrico, los valores correspondientes por fuera de la diagonal serían mayores que en el caso de un barrio menos visitado. Para este trabajo consideraremos que $C$ está fija en el tiempo pero existe la posibilidad de diseñar esta matriz con mayor detalle: por ejemplo se podría ajustar utilizando datos de movilidad que cambien en el tiempo para contemplar cambios que afectarían al contacto entre personas o incluso para incluír los efectos de medidas de control como cuarentenas o restricciones de movilidad. Aquí utilizamos expresiones sencillas para la matriz de contactos porque el objetivo principal es el de la evaluación de la metodología propuesta para asimilar datos en ABMs.

El modelo puede ser extendido añadiendo más atributos a los agentes o representando con mayor detalle los patrones de contacto entre personas. Por ejemplo, sería posible incorporar edades o perfiles sociales para enriquecer la configuración de relaciones entre contactos que sean relevantes para la descripción de la dispersión de la enfermedad. Añadir eventos sociales de gran concurrencia, como por ejemplo lugares de trabajo o escuelas puede ser útil para el modelado de los fenómenos de dispersión masiva (conocidos como eventos \textit{superspreader}). También es posible incluir otras características como las campañas de vacunación o el sugrimiento de nuevas cepas del virus. Incorporando campos a la descripción del estado de los agentes se puede determinar si están vacunados, cuantas dosis recibieron, etc. Por otro lado, se puede indicar con qué variante del virus se contagiaron los agentes infectados.

El ABM subdivide al total de agentes en subpoblaciones de acuerdo a categorías epidemiológicas de manera similar a modelos compartimentales de ecuaciones diferenciales. Sin embargo los ABMs no pueden ser descriptos de manera directa con ecuaciones diferenciales. Cuando el número de agentes es grande las variables agregadas pueden suavizar el efecto de las componentes estocásticas (por ejemplo de los tiempos muestreados de distribuciones Gamma) y como resultado pueden llegar a ser reproducibles con modelos de ecuaciones diferenciales. Los ABMs tienen características, tal como el comportamiento de tener agentes que residen en hogares y con tasas de infección diferenciadas entre contactos domésticos y casuales, para las cuales no está claro como se podrían traducir a un modelo basado en ecuaciones. El modelado a nivel microscópico de los ABMs puede tener consecuencias en la gran escala que pueden no ser fáciles de predecir. A pesar de esto, es importante destacar que los ABMs pueden tener un alto costo computacional lo cual a motivado el uso de modelos de ecuaciones que actúan como sustitutos de un ABM y que pueden ser utilizados para realizar inferencia con bajo costo computacional \citep{Hooten2020}.

\section{Asimilación de datos en ABMs}

\section{Experimentos}
