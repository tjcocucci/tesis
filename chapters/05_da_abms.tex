\chapter{Asimilación de datos en modelos basados en agentes}

\section{Modelos basados en agentes}

A diferencia de los modelos de predeicción con ecuaciones diferenciales, los modelos basados en agentes (conocidos como ABMs por sus siglas en inglés) se basan en un paradigma diferente. Estos modelan de manera explícita las características y comportamiento de individuos o agentes que interactúan. El comportamiento conjunto de la población de agentes se utiliza como modelo de un sistema complejo \citep{Bonabeau2002}. Incluso reglas de interacción simples pueden resultar en que los agentes se organicen de manera autónoma y que el comportamiento colectivo de la población emerja de manera \textit{bottom-up} desde la escala micro del modelado de los individuos \citep{Helbing2012}. En general, los ABMs requieren una gran capacidad computacional ya que las poblaciones de agentes pueden ser muy grandes. Actualmente es factible utilizarlos de manera operacional y existen ejemplos de su aplicación en epidemiología, ecología, economía y sociología \citep{Vynnycky2010, Grimm2005, Tesfatsion2006, Epstein1996}.

Los modelos de agentes tienen dos componentes que los describen: 
\begin{enumerate}
    \item Las descripciones de los agentes
    \item Las reglas de comportamiento e interacción
\end{enumerate}
Los agentes pueden ser vistos como un tipo de dato con diferentes campos de manera que cada individuo se define por el valor de estos atributos. Para completar al modelo se agrega el comportamiento e interacción de los agentes, en general mediante reglas que pueden contener componentes estocásticos. Las acciones que realicen los individuos baje estas disposiciones provocarán cambios en los valores de sus atributos. En principio estos atributos pueden ser cualquier tipo de dato, de manera que los agentes pueden ser programados como tipos de datos compuestos (como estructuras en C) o incluso a través de clases en lenguajes orientados a objetos. Por ejemplo, se pueden usar números reales para representar coordenadas espaciales, variables categóricas para clases sociales, números enteros para edades, etc.

Para un modelar la dispersión de una enfermedad es apropiado construir ABMs epidemiológicos tomando una población de agentes en el que cada uno de ellos representa a una persona (o individuo susceptible a contraer y/o transmitir la enfermedad) \citep{Roche2011}. Utilizando variables categóricas se puede etiquetar a cada agente con su status o categoría epidemiológica (susceptible, infectado, etc.) y estas etiquetas pueden ser cambiadas cuando existe un contacto infeccioso o a medida que la enfermedad se desarolla en el tiempo. El modelado de los contactos entre agentes admite múltiples representaciones, se pueden hacer mezclas al azar, a través de grupos, con redes de contactos o modelando explícitamente la movilidad de los agentes en el espacio. El uso de ABMs para epidemiología tiene la virtud de que permite modelar de una manera explícita, y con gran detalle, características relevantes del sistema que suceden en la micro-escala de la intereacción entre individuos. Por ejemplo, la disminución de la inmunidad puede ser representada en cada individuo a través de un atributo. Es posible modelar de manera natural medidas de control como cuarentenas, reastreo de contactos o efectos de vacunación \cite{Silva2020}. Muchos modelos de ecuaciones diferenciales asumen que dentro de cada compartimento la mezcla entre individuos es homogénea; para ABMs epidemiológicos, al disponer de caracterizaciones de cada individuo, es posible obtener patrones de interacción mas ricos a través de redes de contacto o el mecanismo que resulte más conveniente. De esta manera se pueden capturar efectos difíciles de representar con modelos de ecuaciones diferenciales. A pesar de disponer de información de cada agente, en la simulación con ABMs el interés está en el comportamiento global del sistema. Se suele contar con una función que de alguna manera resuma al estado de la población como un todo a través de un conjunto de variables que llamaremos variables agregadas. En general, si tenemos que la población de agentes es $A$, llamaremos $\v x$ a las variables agregadas y $\phi$ a la función sintetizante que realiza el mapeo, de manera que:
\begin{align}
    \phi (A) = \v x
\end{align}
El comportamiento de las variables agregadas emergerá del comportamiento a nivel individual del ABM. En el caso de modelos epidemiológicos es razonable, por ejemplo, utilizar como resumen el conteo de la cantidad de individuos en cada categoría para saber cuantos susceptibles, infectados, recuperados, etc. hay.

Con el surgimiento de la pandemia de COVID-19 se han desarrollado una gran diversidad de ABMs. Algunos de estos incorporan, entre otras características, estructura de edades y redes de contactos a través de escuelas, casas, lugares de trabajo que permiten modelar de manera más realista las mezclas que dan lugar a los contacios \citep{Kerr2020,Flaxman2020,Simoy2021}. Además de que el aumento de la capacidad de cómputo hace más viable la utilización de ABMs, el gran caudal de datos recolectados a nivel individual a través de dispositivos electrónicos es otro gran aliciente para el incremento en su popularidad. En \cite{Aleta2020} se utilizan datos demográficos y de movilidad para construir las redes de contactos y distribución de hogares en un ABM que permite evaluar los efectos de las intervenciones no-farmacéuticas.

A pesar de la flexibilidad y expresividad que permite el modelado mediante agentes, sigue siendo necesario calibrarlos adecuadamente. En general, el aumento en complejidad viene acompañado de un aumento en el número de parámetros, los cuales no siempre son especificables a través del conocimento que se tenga sobre el sistema. Ha habido avances en el desarrollo de técnicas de inferencia para estimar parámetros en ABMs utilizando observaciones sobretodo orientados a la maximización de la verosimilitud. En particular en \cite{Hooten2020} se discuten una variedad de métodos para maximizar aproximaciones de la verosimilitud usando cómputo bayesiano aproximado (o ABM por \textit{Approximate Bayesian Computation}) junto con técnicas de MCMC (\textit{Markov Chain Monte Carlo}).

Como los ABMs se enmarcan en un contexto de sistemas con evolución temporal parcialmente observados, es posible pensar que la caja de herramientas provista por la asimilación de datos secuencial tiene potencial para realizar las tareas de inferencia. Un trabajo pionero en esta dirección es \cite{Ward2016} en el que se utiliza el EnKF para asimilar datos de cámaras de pisadas en calles Leeds con un modelo de agentes para estudiar el comportamiento del tráfico peatonal. A pesar de obtener resultados satisfactorios, se señala la dificultad asociada a la sensiblidad respecto a parámetros del modelo y la necesidad optimizar el código para modelos de gran tamaño. Como mencionamos anteriormente, el interés no necesariamente está puesto en estado de cada individuo sino en el estado global del sistema o deun conjunto de variables que den una descripción de la población de agentes que se considere relevante. Además, a pesar de que el estado a la escala microscópica determina de manera total al sistema y que el objetivo de la inferencia fuera, idealmente, representar esta escala de la manera más precisa posible, sucede que usualmente las observaciones que se tienen del sistema son de las varables sintetizantes, es decir de la escala macroscópica. Es decir, las observaciones pueden no tener suficiente ``resolución'' como para que se pueda determinar el estado de los atributos de cada agente. Nuestra propuesta, publicada en \cite{Cocucci2022}, es la de asimilar datos sobre las variables agregadas mediante técnicas basadas en ensambles. El mapeo de la escala microscópica a macroscópica está dado por la función sintetizante $\phi$ pero no tenemos una representación para su inversa. De hecho $\phi$ puede no ser inyectiva y por lo tanto no-invertible: muchas configuraciones de los agentes pueden resultar en el mismo estado de las variables agregadas. Esta situación debe considerarse para hacer un emparejamiento entre los estados macroscópicos observados y el estado de los atrbutos de los individuos.

\subsection{Modelo SEIHRD}
