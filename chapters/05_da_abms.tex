\chapter{Asimilación de datos en modelos basados en agentes}

\section{Modelos basados en agentes}

A diferencia de los modelos de predeicción con ecuaciones diferenciales, los modelos basados en agentes (conocidos como ABMs por sus siglas en inglés) se basan en un paradigma diferente. Estos modelan de manera explícita las características y comportamiento de individuos o agentes que interactúan. El comportamiento conjunto de la población de agentes se utiliza como modelo de un sistema complejo \citep{Bonabeau2002}. Incluso reglas de interacción simples pueden resultar en que los agentes se organicen de manera autónoma y que el comportamiento colectivo de la población emerja de manera \textit{bottom-up} desde la escala micro del modelado de los individuos \citep{Helbing2012}. En general, los ABMs requieren una gran capacidad computacional ya que las poblaciones de agentes pueden ser muy grandes. Actualmente es factible utilizarlos de manera operacional y existen ejemplos de su aplicación en epidemiología, ecología, economía y sociología \citep{Vynnycky2010, Grimm2005, Tesfatsion2006, Epstein1996}.

Los modelos de agentes tienen dos componentes que los describen: 
\begin{enumerate}
    \item Las descripciones de los agentes
    \item Las reglas de comportamiento e interacción
\end{enumerate}
Los agentes pueden ser vistos como un tipo de dato con diferentes campos de manera que cada individuo se define por el valor de estos atributos. Para completar al modelo se agrega el comportamiento e interacción de los agentes, en general mediante reglas que pueden contener componentes estocásticos. Las acciones que realicen los individuos baje estas disposiciones provocarán cambios en los valores de sus atributos. En principio estos atributos pueden ser cualquier tipo de dato, de manera que los agentes pueden ser programados como tipos de datos compuestos (como estructuras en C) o incluso a través de clases en lenguajes orientados a objetos. Por ejemplo, se pueden usar números reales para representar coordenadas espaciales, variables categóricas para clases sociales, números enteros para edades, etc.

Para un modelar la dispersión de una enfermedad es apropiado construir ABMs epidemiológicos tomando una población de agentes en el que cada uno de ellos representa a una persona (o individuo susceptible a contraer y/o transmitir la enfermedad) \citep{Roche2011}. Utilizando variables categóricas se puede etiquetar a cada agente con su status o categoría epidemiológica (susceptible, infectado, etc.) y estas etiquetas pueden ser cambiadas cuando existe un contacto infeccioso o a medida que la enfermedad se desarolla en el tiempo. El modelado de los contactos entre agentes admite múltiples representaciones, se pueden hacer mezclas al azar, a través de grupos, con redes de contactos o modelando explícitamente la movilidad de los agentes en el espacio. El uso de ABMs para epidemiología tiene la virtud de que permite modelar de una manera explícita, y con gran detalle, características relevantes del sistema que suceden en la micro-escala de la intereacción entre individuos. Por ejemplo, la disminución de la inmunidad puede ser representada en cada individuo a través de un atributo. Es posible modelar de manera natural medidas de control como cuarentenas, reastreo de contactos o efectos de vacunación \cite{Silva2020}. Muchos modelos de ecuaciones diferenciales asumen que dentro de cada compartimento la mezcla entre individuos es homogénea; para ABMs epidemiológicos, al disponer de caracterizaciones de cada individuo, es posible obtener patrones de interacción mas ricos a través de redes de contacto o el mecanismo que resulte más conveniente. De esta manera se pueden capturar efectos difíciles de representar con modelos de ecuaciones diferenciales. A pesar de disponer de información de cada agente, en la simulación con ABMs el interés está en el comportamiento global del sistema. Se suele contar con una función que de alguna manera resume el 

% Usually, what is of interest is not the particular outcome of the simulation of an individual but rather the resulting state of the system as a whole. Simple interaction rules can yield complex global behavior. In this case, the total number of agents at each category at a given time gives the aggregated representation of the agent-based system. The resulting aggregated variables are not modeled with differential equations: the state of each subpopulation emerges from the individual-based level.

Con el surgimiento de la pandemia de COVID-19 se han desarrollado una gran diversidad de ABMs. Algunos de estos incorporan, entre otras características, estructura de edades y redes de contactos a través de escuelas, casas, lugares de trabajo que permiten modelar de manera más realista las mezclas que dan lugar a los contacios \citep{Kerr2020,Flaxman2020,Simoy2021}. Además de que el aumento de la capacidad de cómputo hace más viable la utilización de ABMs, el gran caudal de datos recolectados a nivel individual a través de dispositivos electrónicos es otro gran aliciente para el incremento en su popularidad. En \cite{Aleta2020} se utilizan datos demográficos y de movilidad para construir las redes de contactos y distribución de hogares en un ABM que permite evaluar los efectos de las intervenciones no-farmacéuticas.

A pesar de la flexibilidad y expresividad que permite el modelado mediante agentes, sigue siendo necesario calibrarlos adecuadamente. En general el aumento en complejidad viene acompañado de un aumento en el número de parámetros los cuales no siempre son especificables a través del conocimento que se tenga sobre el sistema.

% One of the main limitations of ABMs is the need for setting multiple simulation parameters. Recently, there were efforts to develop inference techniques to constrain ABM parameters through available observations. These are mainly focused on obtaining a proxy for the likelihood. In [12], a variety of methods are proposed to calibrate ABMs, for instance, using Approximate Bayesian Computation alongside Markov chain Monte Carlo or approximating the likelihood using an emulator for the ABM.

\subsection{Modelo SEIHRD}
