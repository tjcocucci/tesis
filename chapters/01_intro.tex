En esta tesis exploro algunos de los desafíos asociados a la aplicación de técnicas de asimilación de datos basadas en ensambles sobre modelos epidemiológicos. Por un lado estudio el potencial de utilizar estas técnicas para modelos basados en agentes, en contraste al caso habitual de modelos de ecuaciones diferenciales. Por el otro, abordo el problema de la especificación de los errores observacionales y de modelo en este tipo de sistemas. Para esto expongo algunos métodos basados en el algoritmo EM y propongo una metodología de estimación online de parámetros de la distribución de los errores.

La asimilación de datos comprende un conjunto de técnicas estadísticas que se utilizan para combinar dos fuentes de información distintas sobre el estado de un mismo sistema: pronósticos y observaciones. La disciplina está fuertemente emparentada con la predicción numérica meteorológica. En esta área, se cuenta con modelos matemáticos y computacionales muy complejos y de alta dimensionalidad que informan sobre diversas variables de estado (por ejemplo, temperatura o presión) en diferentes puntos de una grilla espacial potencialmente muy grande. Estos modelos se basan en leyes físicas (por ejemplo, las ecuaciones de Navier-Stokes expresan la conservación del momento y de la masa en en fluidos), y permiten obtener pronósticos. Por otro lado, se tiene otra fuente de información sobre el mismo sistema que consta de las observaciones de diversos instrumentos en estaciones meteorológicas o provenientes de satélites. Ambas fuentes de información son propensas a errores. El error de modelo que comprende nuestro conocimiento limitado de la dinámica del sistema, aproximaciones y errores numéricos. El error observacional incuye la incerteza propia de los instrumentos de medición y el error de representatividad que involucra como se relacionan las observaciones con el estado del sistema (más sobre esto en sección REF). La asimilación de datos busca una es capaz de encontrar una combinación ponderada entre estas fuentes de información, de manera que si sabemos que la incerteza del modelo es menor que la de los datos, la estimación resultante será más fiel al modelo y viceversa. Muchas de las técnicas de asimilación de datos fueron desarrolladas para resolver problemas asociados a la predicción numérica meteorológica pero el campo de apicación es mucho más diverso. Se utilizan, por ejemplo, para navegación aeorespacial \cite{Grewal2010}, predicción sobre reservorios petrolíferos \cite{Aanonsen2009}, oceanografía, detección de incendios forestales, epidemiología, entre otras. (CITE)

[[hablar de manera general y coloquial sobre asimilación de datos y hacer referenica a la formalizacion en el marco teorico]]

[[Modelos epidemiológicos ABMs referencia a especificacion del SEIHRD en metodos]]
Mathematical models of infectious disease transmission have
been in use for over a century (4). These models have been de-
veloped to study the dynamic properties of disease transmission
(5–7), determine the biological characteristics of specific patho-
gens (8, 9), and analyze historical transmission behavior during
documented outbreak events (10).

[[Aplicaciones de DA en epidemiología y referencia a DA+ABM en metodos]]

[[Error observacional y de modelo, metodos batch versus metodos online y referencia a EM en marco teórico y a online EM en metodos]]

