\chapter{Introducción}
En esta tesis exploramos algunos de los desafíos asociados a la aplicación de técnicas de asimilación de datos basadas en ensambles sobre modelos epidemiológicos. Abordamos el problema de la especificación de la incerteza inherente al modelo y las observaciones tanto en un marco general de modelos parcialmente observados como para el caso específico de modelos epidemiológicos basados en ecuaciones diferenciales. Además estudiamos el potencial de utilizar técnicas de asimilación de datos en modelos basados en agentes.

La asimilación de datos comprende un conjunto de técnicas estadísticas que se utilizan para combinar dos fuentes de información distintas sobre el estado de un mismo sistema: pronósticos provenientes de modelos matemáticos y observaciones. La disciplina está fuertemente emparentada con la predicción numérica meteorológica pues gran parte de su desarrollo está orientada a esos fines \citep{Talagrand1987}. En esta área, se cuenta con modelos matemáticos y computacionales muy complejos y de alta dimensionalidad que informan sobre diversas variables de estado (por ejemplo, velocidad, temperatura o presión) en diferentes puntos de una grilla espacial potencialmente muy grande ($10^7 - 10^8$ dimensiones). Estos modelos se basan en leyes físicas (por ejemplo, las ecuaciones de Navier-Stokes que expresan la conservación del momento y de la masa en fluidos), y permiten obtener pronósticos. Por otro lado, se tiene otra fuente de información sobre el mismo sistema que consta de las observaciones de diversos instrumentos en estaciones meteorológicas o provenientes de satélites. Ambas fuentes de información son propensas a errores. El error de modelo que comprende nuestro conocimiento limitado de la dinámica del sistema, aproximaciones y errores numéricos. El error observacional incuye la incerteza propia de los instrumentos de medición y el error de representatividad que involucra como se relacionan las observaciones con el estado del sistema (más sobre esto en la Sección \ref{sec:model_obs_error}). La asimilación de datos apunta a encontrar una combinación ponderada entre estas fuentes de información, de manera que si sabemos que la incerteza del modelo es menor que la de los datos, la estimación resultante será más fiel al modelo y si por el contrario, las observaciones tienen menos error que el pronóstico la estimación será más próxima a los datos.

El filtro de Kalman \citep{Kalman1960, Kalman1961} ocupa un lugar central dentro de las técnicas de asimilación de datos pues es una metodología sencilla que ha sentado las bases para métodos más sofisticados. Este tipo de filtro lineal encontró aplicaciones relevantes en la determinación de órbitas satelitales, navegación de submarinos y aeronaves e incluso de misiones espaciales como la Apollo \citep{Jazwinski1970}. En este tipo de aplicaciones típicamente tenemos que con el fin de estimar la posición y velocidad se utiliza como modelo a las ecuaciones físicas de movimiento mientras que las observaciones provienen de los instrumentos de navegación. Una gran parte del desarrollo de técnicas de asimilación de datos proviene sin embargo del área de las geociencias donde se presentan otro tipo de desafíos como la alta dimensionalidad de los sistemas, observaciones menos precisas y modelos caóticos altamente no-lineales. Por ejemplo, el filtro de Kalman por ensambles \citep{Evensen1994} toma la idea original de Kalman incorpora la idea de representar distribuciones mediante muestras lo cual permite adaptar el problema a situaciones de no-linealidad. La familia de métodos por ensambles pudo competir con los más tradicionales métodos variacionales (3D-VAR y 4D-VAR) que formulan al problema como la minimización de una función de costo y que son utilizados en grandes centros meteorológicos \citep{Kalnay2007}. Otras técnicas más modernas han comenzado a ganar relevancia en el campo: notablemente los filtros de partículas que permiten la representación de distribuciones no Gaussianas han conformado otra gran familia de metodologías de creciente interés \citep{vanLeeuwen2019}. Muchas de las técnicas de asimilación de datos fueron desarrolladas inicialmente para resolver problemas asociados a la predicción numérica meteorológica pero el campo de apicación es mucho más diverso. Se utilizan, por ejemplo, para navegación aeorespacial \cite{Grewal2010}, predicción sobre reservorios petrolíferos \cite{Aanonsen2009}, oceanografía, detección de incendios forestales \citep{Mandel2008}, epidemiología \citep{Shaman2012}, entre otras. En el Capítulo \ref{chp:da} formulamos el problema de la asimilación de datos desde una perspectiva Bayesiana e introducimos las técnicas más relevantes para el desarrollo de nuestro trabajo.
\ma{citar review gral DA. cual?}

Como mencionamos anteriormente, la asimilación de datos tiene en cuenta la incerteza del modelo que genera los pronósticos del sistema tanto como la proveniente de las observaciones. Una especificación errónea de estas cantidades puede causar una performance subóptima de la inferencia pero es habitual que estos errores sean difíciles de identificar. Por lo tanto, existe una variedad de métodos para proveer estimaciones para el error observacional y de modelo \citep{Tandeo2020}. Estos incluyen metodologías basadas en momentos estadísticos y otras que apuntan a la maximización de la verosimilitud entre las cuales hacemos mención del algoritmo EM \citep{Dempster1977}, el cual puede ser implementado en el contexto de asimilación de datos con filtros de Kalman y en particular con filtros basados en ensambles \citep{Tandeo2015}. Como contribución en esta dirección hemos desarrollado una técnica basada en EM pero que tiene la cualidad de ser adaptativa (\textit{online}). Esto significa que a diferencia de su implementación tradicional que procesa los datos en lotes, esta versión procesa cada observación una única vez de manera secuencial. En el Capítulo \ref{chp:error_treatment} introducimos el problema de tratamiento de errores, discutimos algunas de las metodologías más conocidas y finalmente presentamos nuestro algoritmo EM \textit{online} con su deducción teórica y una evaluación experimental de su desempeño.

El marco en el que se suele aplicar la asimilación de datos es el de un sistema que evoluciona temporalmente y es parcialmente observado. A pesar de no ser una de las áreas típicas de aplicación, la epidemiología es un área que puede servise de las herramientas de la asimilación de datos. Desde comienzos del siglo pasado se han utilizado modelos matemáticos para estudiar las dinámicas de transmisión de enfermedades. Los populares modelos compartimentales basados en ecuaciones diferenciales \citep{Kermack1927} han sido adaptados a una gran diversidad de escenarios epidemiológicos. Sin embargo, los ejemplos de aplicación de asimilación de datos sobre estos modelos no son tan abundantes como en otras áreas y la estimación de errores observacionales y de modelo en estos sistemas es infrecuente. En el Capítulo \ref{chp:epi_models} introducimos los elementos básicos del modelado epidemiológico y hacemos una evaluación experimental del EM \textit{online} acoplado con el filtro de Kalman por ensambles aplicado a modelos epidemiológicos compartimentales.

Los modelos epidemiológicos compartimentales están típicamente representados mediante ecuaciones diferenciales y las técnicas de asimilación de datos son en general aplicadas a modelos dinánmicos con esta característica. Sin embargo, se han popularizado, en parte a las posiblidades que abre el aumento del poder de cómputo, los modelos basados en agentes. Estos, en lugar de tomar las variables de interés y representar su comportamiento mediante ecuaciones, simulan una población de individuos a través de un conjunto de reglas de interacción. El comportamiento global del sistema es entonces el resultado del comportamiento colectivo de la totalidad de los agentes. Este paradigma provee gran adaptabilidad y expresividad a estos modelos ya que permiten representar situaciones mediante reglas simples de interrelación entre individuos que pueden ser muy difíciles de reproducir mediante ecuacuiones diferenciales. Con la notable excepción del trabajo de \cite{Ward2016}, no existen en nuestro conocimiento aplicaciones de asimilación de datos sobre modelos de agentes. En el Capítulo \ref{chp:da_abms} damos un marco general de aplicación de asimilación de datos basada en ensambles para modelos basados en agentes y hacemos una evaluación experimental de la metodología sobre un modelo que desarrollamos en base a las características epidemiológicas del COVID-19.
